% Setup -------------------------------

\documentclass[a4paper]{report}
\usepackage[a4paper, total={6in, 10in}]{geometry}
\setcounter{secnumdepth}{3}
\setcounter{tocdepth}{3}

\usepackage{hyperref}
\usepackage{indentfirst}

\usepackage{fancyvrb}
\usepackage{xcolor}

\usepackage{graphicx}

\usepackage{float}

% Encoding
%--------------------------------------
\usepackage[T1]{fontenc}
\usepackage[utf8]{inputenc}
%--------------------------------------

% Portuguese-specific commands
%--------------------------------------
\usepackage[portuguese]{babel}
%--------------------------------------

% Hyphenation rules
%--------------------------------------
\usepackage{hyphenat}
%--------------------------------------

% Capa do relatório

\title{
	Gestão de Grandes Conjuntos de Dados
	\\ \Large{\textbf{2º Trabalho Prático}}
	\\ -
	\\ Mestrado em Engenharia Informática
	\\ Universidade do Minho
}
\author{
	\begin{tabular}{ll}
		\textbf{Grupo nº 8}
		\\
		\hline
		PG41080 & João Ribeiro Imperadeiro
        \\
		PG41081 & José Alberto Martins Boticas
		\\
        PG41091 & Nelson José Dias Teixeira
        \\
        PG41851 & Rui Miguel da Costa Meira
	\end{tabular}
}

\date{\today}

\begin{document}

\begin{titlepage}
    \maketitle
\end{titlepage}

% Índice

\tableofcontents
\listoffigures

% Introdução

\chapter{Introdução} \label{ch:Introduction}
\large {
	Neste trabalho prático é requerida a concretização e avaliação experimental de tarefas de armazenamento e processamento de dados através do uso da ferramenta computacional \textit{Spark} (\textit{batch} e \textit{streaming}).
	Por forma a realizar estas tarefas, são utilizados os dados públicos do \textit{IMDb}, que se encontram disponíveis em:
	\begin{center}
		\textit{\url{https://www.imdb.com/interfaces/}}
	\end{center}

	Para além destes dados, é também utilizado um gerador de \textit{streams}, baseado nos mesmos, que simula uma sequência de votos individuais de utilizadores. Este utensílio foi desenvolvido pelo docente desta unidade curricular e encontra-se disponível na plataforma \textit{Blackboard}.

	Ao longo deste documento vão também ser expostos todos os passos tomados durante a implementação das tarefas pedidas neste projeto, incluindo as decisões tomadas pelos elementos deste grupo a nível de algoritmos e parâmetros de configuração.
	Para além disso são ainda apresentadas todas as instruções que permitem executar e utilizar corretamente os programas desenvolvidos.
	Por fim, na fase final deste manuscrito, são exibidos os objetivos atingidos após a realização das tarefas propostas.

	De salientar também que durante os capítulos que se seguem são identificadas algumas alternativas para concretizar as tarefas indicadas neste trabalho prático.	
}

\chapter{Implementação} \label{ch:Implementation}
\large {
    \section{A} \label{sec:A}
    
	\section{B} \label{sec:B}
		\subsection{B1} \label{subsec:B1}
			\subsubsection{B11} \label{sssec:B11}

		\subsection{B2} \label{subsec:B2}
			\subsubsection{B21} \label{sssec:B21}
			\subsubsection{B22} \label{sssec:B22}

		\subsection{B3} \label{subsec:B3}

\chapter{Conclusão} \label{ch:Conclusion}
\large{
	
}

\appendix
\chapter{Observações} \label{ch:Observations}
\begin{itemize}
    \item Documentação \textit{Java} 8:
    \par \textit{\url{https://docs.oracle.com/javase/8/docs/api/}}
	\item \textit{Maven}:
	\par \textit{\url{https://maven.apache.org/}}
	\item \textit{Apache Spark}:
	\par \textit{\url{https://spark.apache.org/}}
	\item \textit{Docker}:
	\par \textit{\url{https://www.docker.com/}}
\end{itemize}

\end{document}